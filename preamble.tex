
%===============================================  packages  ==================================================
\usepackage[T1]{fontenc}
\usepackage[utf8]{inputenc}
\usepackage{cmbright}
\usepackage[english]{babel}
\usepackage{amsmath,amsfonts,amssymb}
\usepackage{siunitx,units}
\usepackage{calc}
\usepackage{ifthen}
\usepackage{graphicx}
\usepackage[usenames,dvipsnames,table]{xcolor}
\usepackage{pict2e}
\usepackage{tikz}
\usetikzlibrary{calc,arrows,shapes,positioning}
\usepackage{bm} 
\usepackage{array}
\usepackage{multirow}
\usepackage{multicol}
\usepackage{booktabs}
\usepackage{verbatim}
\usepackage{enumitem}
\usepackage{pgfplots}
\pgfplotsset{compat=newest}
\usetikzlibrary{arrows.meta,decorations.markings}

\usepackage{hyperref}
\hypersetup{
pdftitle={Physics 33-104 Poster},
pdfsubject={Speed of sound},
pdfauthor={Tanvi Jakkampudi},
pdfkeywords={poster}
}


\input{arrowsnew}
%============================================================================Wave function
\newcommand\wave{%
  \centering
	\begin{tikzpicture}[]
	      \begin{scope}
        \begin{axis}[width=31.5in,height=13cm,
            hide axis]
          \addplot [color=coolblack,only marks, domain=0.5:11*pi-2, samples=2400, mark size=2,mark=*]
                      ({sin(deg(x+0.3*rand)) + x+0.3*rand}, {rand});
          \addplot [mark=none, color=red, line width=1.5mm, domain=1:11*pi-1, smooth, samples=1000] {cos(deg(\x))*-1};
          \draw[line width=1mm,rounded corners=2cm, shading=axis, left color=rose!70, right color=Bittersweet!50, middle color=lemon!50, shading angle=40, draw=black!90, opacity=0.35] (-1,-1) rectangle (11*pi+1,1);
          \draw[fill=coolblack!90,very thick, draw=coolblack!90] (0,-1) rectangle (1,1);
          \draw[rounded corners=2cm,fill=coolblack!90,very thick, draw=coolblack!90] (-1,-1) rectangle node[xshift=0.6cm,above,font=\large,rotate=90, text=white]{Speaker} (1,1);
          \coordinate (box1) at (-1,0);
          \coordinate (lend) at (-1,1);
          \draw[fill=coolblack!90,very thick, draw=coolblack!90] (11*pi-1,-1) rectangle (11*pi,1);
          \draw[rounded corners=2cm,fill=coolblack!90,ultra thick, draw=coolblack!90] (11*pi-1,-1) rectangle node[xshift=-0.6cm,above,font=\large,rotate=-90, text=white]{Microphone} (11*pi+1,1);
          \coordinate (box2) at (11*pi+1,0);
          \coordinate (rend) at (11*pi+1,1);
          \coordinate (max1) at ({pi},-1.1);
          \coordinate (wl1) at ({pi},1.1);
          \coordinate (max2) at ({3*pi},-1.1);
          \coordinate (wl2) at ({3*pi},1.1);
          \coordinate (max3) at ({5*pi},-1.1);
          \coordinate (min1) at ({6*pi},-1.1);
          \coordinate (min2) at ({8*pi},-1.1);
          \coordinate (min3) at ({10*pi},-1.1);
        \end{axis}
        \tikzset{myptr/.style={decoration={markings,mark=at position 1 with %
    {\arrow[arrowhead=1cm,>=tonew]{>}}},postaction={decorate}}};
    	\tikzset{myptr2/.style={decoration={markings,mark=at position 1 with %
    {\arrow[arrowhead=1cm,>=tonew]{<}}},postaction={decorate}}};
        \node[text=coolblack, below] at (max1) {\normalsize\bf Compression};
        \node[text=coolblack, below] at (max2) {\normalsize\bf High~pressure};
        \node[text=coolblack, below] at (min2) {\normalsize\bf Rarefaction};
        \node[text=coolblack, below] at (min3) {\normalsize\bf Low~pressure};
        \draw[draw=coolblack, {Straight Barb[line width=1mm,scale=2,length=0pt,width=8mm]<}-{> Straight Barb[line width=1mm,scale=2,length=0pt,width=8mm]}, line width=1mm]
              ([yshift=0cm]lend |- current axis.north)
              -- node[above, text=coolblack, yshift=0.4cm]{\normalsize\bf Length ($\bm L$) of the system = 1.215~m} ([yshift=0cm]rend |- current axis.north);
        \node(bin1)[yshift=2.25cm,xshift=-9cm,circle,minimum size=5cm,inner sep=10pt, text width=4.5cm, fill=Purple!70, text=white, align=center] at (box1 |- current axis.west) {\bf ($\bm f$) generated by function generator};
        \draw[->, red, line width=1.5mm] (bin1.north) to [out=45,in=115] (box1.center);
        \node(bin2)[xshift=3.5cm,yshift=2.25cm,circle,minimum size=5cm,inner sep=10pt, text width=4.5cm, fill=brown!50, text=coolblack, align=center] at (box2 |- current axis.east) {\bf ($\bm n$)\\measured with Oscilloscope};
        \draw[->, red, line width=1.5mm] (box2.center) to [out=-75,in=270] (bin2.south);
        \end{scope}
      
    \end{tikzpicture}
  }
%===============================================  core layout  ===============================================
\usepackage[paperwidth=40in,
paperheight=30in,
textwidth=40in-4cm,
textheight=30in-4cm,
centering]{geometry}
\pagenumbering{gobble}
\setlength{\parindent}{0cm}
\setlength{\parskip}{.5cm}
\linespread{1.1}
%===============================================  font setup  ================================================

% \renewcommand*\rmdefault{ppl}
\AtBeginDocument{\renewcommand{\Huge}{\fontsize{85pt}{102pt}\selectfont}
	\renewcommand{\huge}{\fontsize{56pt}{67pt}\selectfont}
	\renewcommand{\LARGE}{\fontsize{48pt}{57pt}\selectfont}
	\renewcommand{\Large}{\fontsize{42pt}{50pt}\selectfont}
	\renewcommand{\large}{\fontsize{36pt}{43pt}\selectfont}
	\renewcommand{\small}{\fontsize{18pt}{22pt}\selectfont}
	\renewcommand{\footnotesize}{\fontsize{16pt}{20pt}\selectfont}
	\renewcommand{\scriptsize}{\fontsize{14pt}{16pt}\selectfont}
	\renewcommand{\tiny}{\fontsize{12pt}{14pt}\selectfont} 
	\renewcommand{\normalsize}{\fontsize{24pt}{29pt}\selectfont}
	\normalsize}

%===============================================  commands  ==================================================

%==================================  hacks  ==============================================
\newcommand{\tikzwidth}{\textwidth-\pgflinewidth}
\newcommand{\define}[2]{\newlength{#1}\setlength{#1}{#2}}
\newcommand{\vlinespace}{\vspace{0.167\baselineskip}}
\renewcommand{\textbullet}{$\bullet$}

%==================================  table commands  =====================================
\newcolumntype{C}[1]{>{\centering\let\newline\\\arraybackslash\hspace{0pt}}m{#1}}

%=====================================  spaces and splits  =====================
\newcommand{\matrixnewline}{\\[0.6em]}
\newcommand{\vtiny}{\vspace{.5cm}}
\newcommand{\vsmall}{\vspace{1cm}}
\newcommand{\vmedium}{\vspace{1.5cm}}
\newcommand{\vlarge}{\vspace{3cm}}

%=====================================  math letters  ==========================
\newcommand{\Chi}{\mathcal{X}}
\newcommand{\domain}{\mathcal{D}}
\newcommand{\F}{\mathbb{F}}
\newcommand{\gl}{\mathfrak{gl}}
\newcommand{\N}{\mathbb{N}}
\newcommand{\powerset}{\mathcal{P}}
\newcommand{\proj}{\mathbb{P}}
\newcommand{\Q}{\mathbb{Q}}
\newcommand{\R}{\mathbb{R}}
\newcommand{\range}{\mathcal{R}}
\newcommand{\Z}{\mathbb{Z}}

%=====================================  math symbols  ==========================
\newcommand{\actson}{\,\rotatebox[origin=c]{180}{$\circlearrowright$}\,}
\newcommand{\bi}{\longleftrightarrow}
\newcommand{\fall}{\,\,\,\forall\,}
\newcommand{\idealof}{\setlength{\unitlength}{1ex}\hspace{0.5ex}\begin{picture}(2,2)(0,0.3)\put(0.9,0.9){\circle{1.8}}\polygon(0.06,0.9)(1.459,1.527)(1.459,0.273)\end{picture}\hspace{0.5ex}}
\renewcommand{\iff}{\Leftrightarrow}
\newcommand{\lto}{\longrightarrow}
\newcommand{\normalin}{\vartriangleleft}
\newcommand{\onto}{\twoheadrightarrow}
\newcommand{\onetoone}{\hookrightarrow}
\newcommand{\subgroupof}{\leqslant}
\newcommand{\then}{\Rightarrow}
\newcommand{\vect}[1]{\overrightarrow{#1}}
\newcommand{\xbi}[1]{\overset{#1}{\longleftrightarrow}}
\newcommand{\xfrom}[1]{\overset{#1}{\longleftarrow}}
\newcommand{\xto}[1]{\overset{#1}{\longrightarrow}}


%=====================================  math text  =============================
\newcommand{\Aut}{\text{Aut}}
\newcommand{\cl}{\text{cl}}
\newcommand{\class}{\text{class}}
\newcommand{\im}{\text{im}}
\newcommand{\trace}{\text{trace}}

%=====================================  math delimiters  =======================
\newcommand{\da}[1]{\langle\!\langle{#1}\rangle\!\rangle}
\newcommand{\of}[1]{\!\left({#1}\right)}
\newcommand{\p}[1]{\left({#1}\right)}
\newcommand{\set}[1]{\lbrace {#1} \rbrace}


%=====================================  plain text  ============================
\newcommand{\bs}{\textbackslash}
\newcommand{\degree}{$^\circ$}
\newcommand{\sub}[1]{$_\text{#1}$}
\newcommand{\super}[1]{$^\text{#1}$}


%=====================================  proofs  ================================
\newcommand{\proof}{\noindent\textbf{Proof.} }
\newcommand{\qed}{\hspace*{\fill} $\square$}


%=====================================  entries  ===============================
\newcommand{\ebf}[1]{\emph{\textbf{#1}}}
\newcommand{\pbullet}[1]{\noindent\makebox[1cm]{\textbullet}\parbox[t]{\textwidth-1cm}{{#1}}\vlinespace\strut}
\newcommand{\pbulletb}[1]{\noindent\makebox[2cm]{\hspace*{1cm}\textbullet}\parbox[t]{\textwidth-2cm}{{#1}}\vlinespace\strut}
\newcommand{\pbulletc}[1]{\noindent\makebox[3cm]{\hspace*{2cm}\textbullet}\parbox[t]{\textwidth-3cm}{{#1}}\vlinespace\strut}
\newcommand{\pbulletd}[1]{\noindent\makebox[4cm]{\hspace*{3cm}\textbullet}\parbox[t]{\textwidth-4cm}{{#1}}\vlinespace\strut}
\newcommand{\pcite}[2]{\noindent\makebox[4.5cm][l]{{#1}}\parbox[t]{\textwidth-4.5cm}{{#2}\strut\vsmall}}
\newcommand{\pheading}[1]{\textbf{#1}}
\newcommand{\pheadingb}[1]{\textbf{\large{#1}}}
\newcommand{\pheadingc}[1]{\textbf{\Large{#1}}}
\newcommand{\pheadingd}[1]{\textbf{\LARGE{#1}}}
\newcommand{\definition}[2]{\noindent\textbf{{#1}.} {#2}}
\newcommand{\mathalign}[1]{\noindent\begin{flalign*}{#1}\end{flalign*}}
\newcommand{\mathalignb}[2]{\noindent\parbox{0.5\textwidth}{\mathalign{#1}}\parbox{0.5\textwidth}{\mathalign{#2}}}
\newcommand{\mono}[1]{\texttt{#1}}
\newcommand{\ndefinition}[1]{\noindent\textbf{Definition \stepcounter{D}\arabic{D}.} {#1}}
\newcommand{\theorem}[2]{\noindent\textbf{{#1}.} \emph{#2}\vsmall}
\newcommand{\ntheorem}[1]{\noindent\textbf{Theorem \stepcounter{T}\arabic{T}.} \emph{#1}}
\newcommand{\prebullet}{\noindent\hspace*{-1cm}\makebox[1cm]{\textbullet}}
\AtBeginDocument{\newcounter{T}\newcounter{D}}

%=====================================  poster settings  ===========================================
\definecolor{coolblack}{rgb}{0.0, 0.18, 0.39}
\definecolor{lemon}{rgb}{1.0, 1.0, 0.13}
\definecolor{rose}{rgb}{0, 0.51, 0.5}
\colorlet{backgroundColor}{coolblack!60}

\define{\titleHeight}{11cm}

\define{\logoHeight}{8cm}

\define{\footHeight}{14cm}

\define{\mainMargin}{2cm}

\define{\sectionSideMargin}{1cm}

\define{\sectionInteriorMargin}{1cm}

\define{\sectionTextMargin}{1.25cm}

%=====================================  poster commands  =======================
\define{\pindent}{\parindent}
\define{\pskip}{\parskip}
\define{\footWidth}{(\textwidth-\mainMargin*2)}
\define{\footTextWidth}{\footWidth-\sectionTextMargin*2}
\define{\sectionWidth}{(\textwidth-\mainMargin*2-\sectionSideMargin*2-\sectionInteriorMargin*2)/3}
\define{\sectionTextWidth}{\sectionWidth-\sectionTextMargin*2}
\define{\sectionHeight}{(\textheight-\mainMargin*4-\titleHeight-\footHeight-\sectionInteriorMargin)/2}


\tikzset{below right, sectionBoxHeading/.style={draw=CadetBlue, rectangle, minimum height=3cm, minimum width=\sectionTextWidth, inner ysep=.5cm, line width=1.5mm,rounded corners=.5cm, shading=axis, left color=backgroundColor!50, right color=backgroundColor!50, middle color=white, shading angle=30}}
\tikzset{sectionBox/.style={draw=CadetBlue!50, line width=2mm, rounded corners=1cm, fill=White}}
\tikzset{topLeft/.style={below right,inner sep=0cm}}
\tikzset{topRight/.style={below left,inner sep=0cm}}
\tikzset{bottomLeft/.style={above right,inner sep=0cm}}
\tikzset{bottomRight/.style={above left,inner sep=0cm}}


\newcommand{\poster}[1]{
	\centering
	\begin{tikzpicture}[]
	\draw[draw=none, use as bounding box] (0,0) rectangle (\textwidth,\textheight);
	\draw[shade, top color=coolblack!90, bottom color=coolblack!30, shading angle=25, middle color=RoyalBlue!25, draw=none] (0,0) rectangle (\textwidth,\textheight);
 	\draw[line width=2mm, line cap=rounded, draw=coolblack] (0,0) -- (\textwidth,0);
	#1\end{tikzpicture}}


\newcommand{\ptitle}[4][]{
	\draw[draw=Bittersweet!20, line width=2mm, rounded corners=1cm, shading=axis, left color=Bittersweet!20, right color=rose!60, middle color=white, shading angle=50] (\mainMargin,\textheight-\mainMargin) rectangle (\textwidth-\mainMargin,\textheight-\mainMargin-\titleHeight);
	\ifthenelse{\equal{#1}{}}{} {
		\node[bottomLeft] at(\mainMargin+1cm,\textheight-\mainMargin-\titleHeight+0.5cm) {\includegraphics[height=\logoHeight]{#1}};
		\node[bottomRight] at(\textwidth-\mainMargin-1cm,\textheight-\mainMargin-\titleHeight+0.5cm) {\includegraphics[height=\logoHeight]{#1}};}
	\node[right,inner sep=0cm] at(\mainMargin+1cm,\textheight-\mainMargin-\titleHeight/2) {\begin{minipage}{\textwidth-\mainMargin*2-2cm}
			\centering
			{\Huge\bf\color{coolblack} #2}
			\\\vmedium
			{\huge\bf\color{Bittersweet}#3}
			\\\vsmall
			{\huge \color{coolblack}#4}
			
	\end{minipage}};}


\newcommand{\psection}[3]{
	\ifthenelse{\equal{#1}{1} \OR \equal{#1}{topleft}}{
		\coordinate (a) at(\mainMargin+\sectionSideMargin,\textheight-\mainMargin*2-\titleHeight);}{}
	\ifthenelse{\equal{#1}{2} \OR \equal{#1}{topmiddle}}{
		\coordinate (a) at(\mainMargin+\sectionSideMargin+\sectionWidth+\sectionInteriorMargin,\textheight-\mainMargin*2-\titleHeight);}{}
	\ifthenelse{\equal{#1}{3} \OR \equal{#1}{topright}}{
		\coordinate (a) at(\mainMargin+\sectionSideMargin+\sectionWidth*2+\sectionInteriorMargin*2,\textheight-\mainMargin*2-\titleHeight);}{}
	\ifthenelse{\equal{#1}{4} \OR \equal{#1}{bottomleft}}{
		\coordinate (a) at(\mainMargin+\sectionSideMargin,\textheight-\mainMargin*2-\titleHeight-\sectionHeight-\sectionInteriorMargin);}{}
	\ifthenelse{\equal{#1}{5} \OR \equal{#1}{bottommiddle}}{
		\coordinate (a) at(\mainMargin+\sectionSideMargin+\sectionWidth+\sectionInteriorMargin,\textheight-\mainMargin*2-\titleHeight-\sectionHeight-\sectionInteriorMargin);}{}
	\ifthenelse{\equal{#1}{6} \OR \equal{#1}{bottomright}}{
		\coordinate (a) at(\mainMargin+\sectionSideMargin+\sectionWidth*2+\sectionInteriorMargin*2,\textheight-\mainMargin*2-\titleHeight-\sectionHeight-\sectionInteriorMargin);}{}
	\draw[sectionBox] (a) rectangle ($(a) + (\sectionWidth,-\sectionHeight)$);
	\ifthenelse{\equal{#2}{}}{
		\node[topLeft] at($(a) + (\sectionTextMargin,-\sectionTextMargin)$) {\begin{minipage}{\sectionTextWidth}
				\setlength{\parskip}{\pskip}\setlength{\parindent}{\pindent}#3 \end{minipage}};}
	{\node[sectionBoxHeading] at($(a) + (\sectionTextMargin,-1cm)$) (b) {\parbox{\sectionTextWidth-2cm}{\centering\large\bf\color{coolblack} #2}};
		\node[below=of b, yshift=.5cm] {\begin{minipage}{\sectionTextWidth}
				\setlength{\parskip}{\pskip}\setlength{\parindent}{\pindent} #3 \end{minipage}};}}


\newcommand{\pfoot}[2]{
	\ifthenelse{\equal{#1}{1} \OR \equal{#1}{left}}{
		\coordinate (a) at(\mainMargin,\mainMargin+\footHeight+0.25cm);}{}
	\ifthenelse{\equal{#1}{2} \OR \equal{#1}{right}}{
		\coordinate (a) at(\mainMargin*2+\footWidth,\footHeight);}{}
	  \draw[shading=axis, left color=backgroundColor!50, right color=backgroundColor!50, middle color=white, shading angle=30, draw=none, line width=2mm, rounded corners=1cm, fill opacity=0] (a) rectangle ($(a) + (\footWidth,-\footHeight-0.5cm)$);
	\node[topLeft] at($(a) + (\sectionTextMargin,-\sectionTextMargin+0.5cm)$) {\begin{minipage}{\footTextWidth}
			#2 \end{minipage}};
			}

